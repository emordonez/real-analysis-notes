% MATH PACKAGES
\usepackage{amsmath}
\usepackage{amssymb}
\usepackage{amsthm}
\usepackage{mathtools}
\usepackage{mathrsfs}
\usepackage{bm}
\usepackage{faktor}
\usepackage{nicefrac}

% Blackboard bold shortcuts
\newcommand{\N}{\mathbb{N}}
\newcommand{\Z}{\mathbb{Z}}
\newcommand{\Q}{\mathbb{Q}}
\newcommand{\R}{\mathbb{R}}
\newcommand{\C}{\mathbb{C}}

% Changes to default symbols and letters
\let\implies\Rightarrow
\let\impliedby\Leftarrow
\let\iff\Leftrightarrow
\let\epsilon\varepsilon
\let\phi\varphi
\renewcommand{\vec}[1]{\bm{#1}}
\renewcommand\qedsymbol{$\blacksquare$}

% Math operator declarations and commands
\DeclareMathOperator{\Ima}{Im}
\DeclareMathOperator{\dis}{d}
\DeclareMathOperator{\Lim}{Lim}
\DeclareMathOperator{\interior}{Int}

\DeclarePairedDelimiter\abs{\lvert}{\rvert}
\makeatletter
\let\oldabs\abs
\def\abs{\@ifstar{\oldabs}{\oldabs*}}

\DeclarePairedDelimiter\set{\{}{\}}
\makeatletter
\let\oldset\set
\def\set{\@ifstar{\oldset}{\oldset*}}

\newcommand{\eqclass}[1]{\left[ #1 \right]}
\newcommand{\mult}{{}\cdot{}}
\newcommand{\Mod}[1]{\ (\mathrm{mod}\ #1)}


% FORMATTING PACKAGES
\usepackage[left=1.5in,right=1.5in,top=1in,bottom=1in]{geometry}
% \usepackage{parskip}
\usepackage{textcomp}
\usepackage{eqparbox}
\usepackage{float}
\usepackage{graphicx}
\usepackage{xcolor}
\usepackage{enumitem}
\usepackage{emptypage}
\usepackage{microtype}
\usepackage{tabularx}
\usepackage{todonotes}
\setlength{\marginparwidth}{2cm}
\usepackage[pdfpagelabels]{hyperref}
\usepackage[capitalise,noabbrev]{cleveref}
\usepackage{nameref}

% Horizontal rule
\newcommand\hr{
    \begin{center}
        \noindent\rule[0.5ex]{0.975\linewidth}{0.5pt}
    \end{center}
}


% ENVIRONMENTS
\usepackage[most]{tcolorbox}
\tcbuselibrary{theorems, breakable}
\theoremstyle{definition}

\tcbset{
    thmbox/.style={
        enhanced,
        breakable,
        sharp corners,
        borderline west={2pt}{0pt}{#1},
        boxrule=0pt,
        fonttitle=\sffamily\bfseries,
        coltitle=black,
        colframe=#1,
        colback=white!95!#1,
        colbacktitle=white!75!#1,
        bottomtitle=-0.5pt,
        parbox=false,
        before skip=1.5em,
        after skip=1.5em
    }
}

\newtcbtheorem[number within=chapter]{definition}{Definition}{
    label type=definition,
    thmbox=blue!75!white
}{def}
\newtcbtheorem[number within=chapter]{theorem}{Theorem}{
    label type=theorem,
    thmbox=red!75!white
}{thm}
\newtcbtheorem[number within=chapter]{lemma}{Lemma}{
    label type=lemma,
    thmbox=red!75!white
}{lem}
\newtcbtheorem[number within=chapter]{proposition}{Proposition}{
    label type=proposition,
    thmbox=red!75!white
}{prop}
\newtcbtheorem[number within=chapter]{example}{Example}{
    label type=ex,
    enhanced jigsaw,
    breakable,
    sharp corners,
    borderline west={2pt}{0pt}{black!75!white},
    boxrule=0pt,
    fonttitle=\sffamily\bfseries,
    coltitle=black,
    attach title to upper={\quad},
    parbox=false,
    before skip=1.5em,
    after skip=1.5em
}{ex}
\newtcolorbox{remark}{
    enhanced jigsaw,
    breakable,
    sharp corners,
    borderline west={2pt}{0pt}{black!75!white},
    boxrule=0pt,
    title=Remark,
    fonttitle=\sffamily\bfseries,
    coltitle=black,
    attach title to upper={\quad},
    parbox=false,
    before skip=1.5em,
    after skip=1.5em
}

\crefname{def}{definition}{definitions}
\crefname{thm}{theorem}{theorems}
\crefname{lem}{lemma}{lemmas}
\crefname{prop}{proposition}{propositions}
\crefname{ex}{example}{examples}

% Fix theorem spacing
\makeatletter
\def\thm@space@setup{%
  \thm@preskip=\parskip \thm@postskip=0pt
}

% Exercises
\newenvironment{exercises}
    {\begin{enumerate}[font=\sffamily\bfseries]}
    {\end{enumerate}}


% TITLE AND SECTION STYLING PACKAGES
\usepackage{titling}
\usepackage{titleps}
\usepackage{sectsty}

% Headers and fonts
\newpagestyle{main}[\sffamily]{
    \setheadrule{.7pt}%
    \sethead{\chaptername\ \thechapter}{}{\chaptertitle}
    \setfoot{}{\thepage}{}
}

% Set title page and section headings to sans serif
\pretitle{\begin{center}\LARGE\sffamily}
\posttitle{\par\end{center}\vskip 0.5em}
\preauthor{\begin{center}
    \large\sffamily \lineskip 0.5em%
    \begin{tabular}[t]{c}}
\postauthor{\end{tabular}\par\end{center}}

\allsectionsfont{\sffamily}
\renewcommand{\abstractname}{\textsf{Introduction}}


% C'est moi!
\author{Eric M. Ordo\~nez}
