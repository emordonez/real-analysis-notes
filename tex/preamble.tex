% MATH PACKAGES ===============================================================
\usepackage{amsmath}
\usepackage{amssymb}
\usepackage{amsthm}
\usepackage{mathtools}
\usepackage{mathrsfs}
\usepackage{bm}
\usepackage{faktor}
\usepackage{nicefrac}

% Blackboard bold shortcuts
\newcommand{\N}{\mathbb{N}}
\newcommand{\Z}{\mathbb{Z}}
\newcommand{\Q}{\mathbb{Q}}
\newcommand{\R}{\mathbb{R}}
\newcommand{\C}{\mathbb{C}}

% Changes to default symbols and letters
\let\implies\Rightarrow
\let\impliedby\Leftarrow
\let\iff\Leftrightarrow
\let\epsilon\varepsilon
\let\phi\varphi
\renewcommand{\vec}[1]{\bm{#1}}
\renewcommand{\qedsymbol}{$\blacksquare$}

% Math operator declarations and commands
\DeclareMathOperator{\dis}{d}
\DeclareMathOperator{\Lim}{Lim}
\DeclareMathOperator{\interior}{Int}

\makeatletter
\DeclarePairedDelimiter\abs{\lvert}{\rvert}
\let\oldabs\abs
\def\abs{\@ifstar{\oldabs}{\oldabs*}}

\DeclarePairedDelimiter\setc{\{}{\}}
\let\oldset\setc
\def\setc{\@ifstar{\oldset}{\oldset*}}

\newcommand{\set}[2][\@empty]{%
    \ifx\@empty#1
        \setc{\,#2\,}
    \else
        \setc{\,#1 \;\middle|\; #2\,}
    \fi
}
\makeatother

\newcommand{\eqclass}[1]{\left[ #1 \right]}
\newcommand{\mult}{{}\cdot{}}
\newcommand{\Mod}[1]{\ (\mathrm{mod}\ #1)}


% FORMATTING PACKAGES =========================================================
\usepackage[top=1in,right=1.5in,bottom=1in,left=1.5in,asymmetric]{geometry}
\setlength{\marginparwidth}{2cm}
% \usepackage{parskip}
\usepackage{textcomp}
\usepackage{eqparbox}
\usepackage{float}
\usepackage{graphicx}
\usepackage{xcolor-material}
\usepackage{enumitem}
\usepackage{emptypage}
\usepackage{microtype}
\usepackage{tabularx}
\usepackage[pdfpagelabels]{hyperref}
\usepackage[capitalise,noabbrev]{cleveref}
\usepackage{nameref}

% Horizontal rule
\newcommand\hr{
    \begin{center}
        \noindent\rule[0.5ex]{0.975\linewidth}{0.5pt}
    \end{center}
}


% ENVIRONMENTS ================================================================
\usepackage[most]{tcolorbox}
\tcbuselibrary{breakable, theorems}

\ExplSyntaxOn
\NewDocumentCommand{\betternewtcbtheorem}{O{}mmmm}{%
    \newtcbtheorem[#1]{#2inner}{#3}{#4}{#5}
    \NewDocumentEnvironment{#2}{O{}}{%
        \keys_set:nn { hushus/tcb } { ##1 }
        \hushus_tcb_begin:nVV {#2inner} \l__hushus_tcb_title_tl \l__hushus_tcb_label_tl
    }{%
        \end{#2inner}
    }
    \cs_if_exist:cF { c@#5 } { \newcounter{#5} }
}
\cs_new_protected:Nn \hushus_tcb_begin:nnn {%
    \begin{#1}{#2}{#3}}
\cs_generate_variant:Nn \hushus_tcb_begin:nnn { nVV }
\keys_define:nn { hushus/tcb }{%
    title .tl_set:N = \l__hushus_tcb_title_tl,
    label .tl_set:N = \l__hushus_tcb_label_tl,
}
\ExplSyntaxOff

\tcbset{
    thmtitle/.style={
        attach title to upper=\\[0.5em]
    },
    remtitle/.style={
        attach title to upper=\quad
    },
    thmbox/.style 2 args={
        enhanced jigsaw,
        breakable,
        sharp corners,
        skin=bicolor,
        borderline west={3pt}{0pt}{#1},
        boxrule=0pt,
        fonttitle=\sffamily\bfseries,
        description font=\sffamily\mdseries,
        description delimiters={(}{)},
        description color=#1,
        separator sign={\ },
        coltitle=#1,
        colframe=#1,
        colback=#2,
        colbacklower=white,
        parbox=false,
        before skip=1.5em,
        after skip=1.5em
    }
}

\betternewtcbtheorem[number within=chapter]{definition}{Definition}{
    thmtitle,
    label type=definition,
    thmbox={MaterialBlue700}{MaterialBlue50},
}{def}
\betternewtcbtheorem[number within=chapter]{theorem}{Theorem}{
    thmtitle,
    label type=theorem,
    thmbox={MaterialPurple700}{MaterialPurple50}
}{thm}
\betternewtcbtheorem[number within=chapter]{lemma}{Lemma}{
    thmtitle,
    label type=lemma,
    thmbox={MaterialDeepOrange700}{MaterialDeepOrange50}
}{lem}
\betternewtcbtheorem[number within=chapter]{proposition}{Proposition}{
    thmtitle,
    label type=proposition,
    thmbox={MaterialDeepOrange700}{MaterialDeepOrange50}
}{prop}
\betternewtcbtheorem[number within=chapter]{example}{Example}{
    remtitle,
    label type=example,
    thmbox={MaterialGrey700}{MaterialGrey50}
}{ex}
\newtcolorbox{remark}{
    title=Remark,
    remtitle,
    thmbox={MaterialGreen700}{MaterialGreen50}
}

\crefname{def}{definition}{definitions}
\crefname{thm}{theorem}{theorems}
\crefname{lem}{lemma}{lemmas}
\crefname{prop}{proposition}{propositions}
\crefname{ex}{example}{examples}

% Fix theorem spacing
\makeatletter
\def\thm@space@setup{%
    \thm@preskip=\parskip \thm@postskip=0pt
}
\makeatother

% Exercises
\newenvironment{exercises}
    {\begin{enumerate}[font=\sffamily\bfseries]}
    {\end{enumerate}}


% FIGURES =====================================================================
\usepackage{caption}
\usepackage{import}
\usepackage{pdfpages}
\usepackage{transparent}
    
\newcommand{\incfig}[1]{%
    % \def\svgwidth{\columnwidth}
    \import{../figures/}{#1.pdf_tex}
}
\captionsetup{
    format=plain,
    labelsep=quad,
    labelfont={sf,bf},
    font=sf
}
\numberwithin{figure}{chapter}


% TITLE AND SECTION STYLING PACKAGES ====================================================
\usepackage{etoolbox}
\usepackage{titling}
\usepackage{titleps}
\usepackage{sectsty}

% Headers
\newpagestyle{main}[\sffamily]{
    \sethead
    [\chaptername\ \thechapter][\chaptertitle][\bfseries\thepage]%
    {\S\thesection}{\sectiontitle}{\bfseries\thepage}
}
\patchcmd{\tableofcontents}{plain}{empty}{}{}
\patchcmd{\part}{plain}{empty}{}{}
\patchcmd{\chapter}{plain}{empty}{}{}

% Sans serif titles and section headings
\pretitle{\begin{center}\LARGE\sffamily}
\posttitle{\par\end{center}\vskip 0.5em}
\preauthor{\begin{center}
    \large\sffamily \lineskip 0.5em%
    \begin{tabular}[t]{c}}
\postauthor{\end{tabular}\par\end{center}}

\allsectionsfont{\sffamily}
\sectionfont{\sffamily\sectionrule{0pt}{0pt}{-1ex}{1pt}}
\renewcommand{\abstractname}{\textsf{Introduction}}


% TABLE OF CONTENTS ===========================================================
% \usepackage{tocloft}
% \renewcommand{\cfttoctitlefont}{\sffamily}
% \renewcommand{\cftpartfont}{\sffamily}
% \renewcommand{\cftpartpagefont}{\sffamily}
% \renewcommand{\cftchapfont}{\sffamily}
% \renewcommand{\cftchappagefont}{\sffamily}
% \renewcommand{\cftsecfont}{\sffamily}
% \renewcommand{\cftsecpagefont}{\sffamily}
% \renewcommand{\cftsubsecfont}{\sffamily}
% \renewcommand{\cftsubsecpagefont}{\sffamily}


% C'est moi!
\author{Eric M. Ordo\~nez}
