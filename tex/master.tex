% !TeX program = lualatex
% !TeX encoding = utf-8

\documentclass[11pt,twoside,notitlepage]{report}
\usepackage{subfiles}
% MATH PACKAGES
\usepackage{amsmath}
\usepackage{amssymb}
\usepackage{amsthm}
\usepackage{mathtools}
\usepackage{mathrsfs}
\usepackage{bm}
\usepackage{faktor}
\usepackage{nicefrac}

% Blackboard bold shortcuts
\newcommand{\N}{\mathbb{N}}
\newcommand{\Z}{\mathbb{Z}}
\newcommand{\Q}{\mathbb{Q}}
\newcommand{\R}{\mathbb{R}}
\newcommand{\C}{\mathbb{C}}

% Changes to default symbols and letters
\let\implies\Rightarrow
\let\impliedby\Leftarrow
\let\iff\Leftrightarrow
\let\epsilon\varepsilon
\let\phi\varphi
\renewcommand{\vec}[1]{\bm{#1}}
\renewcommand\qedsymbol{$\blacksquare$}

% Math operator declarations and commands
\DeclareMathOperator{\Ima}{Im}
\DeclareMathOperator{\dis}{d}
\DeclareMathOperator{\Lim}{Lim}
\DeclareMathOperator{\interior}{Int}

\DeclarePairedDelimiter\abs{\lvert}{\rvert}
\makeatletter
\let\oldabs\abs
\def\abs{\@ifstar{\oldabs}{\oldabs*}}

\DeclarePairedDelimiter\set{\{}{\}}
\makeatletter
\let\oldset\set
\def\set{\@ifstar{\oldset}{\oldset*}}

\newcommand{\eqclass}[1]{\left[ #1 \right]}
\newcommand{\mult}{{}\cdot{}}
\newcommand{\Mod}[1]{\ (\mathrm{mod}\ #1)}


% FORMATTING PACKAGES
\usepackage[left=1.5in,right=1.5in,top=1in,bottom=1in]{geometry}
% \usepackage{parskip}
\usepackage{textcomp}
\usepackage{eqparbox}
\usepackage{float}
\usepackage{graphicx}
\usepackage{xcolor}
\usepackage{enumitem}
\usepackage{emptypage}
\usepackage{microtype}
\usepackage{tabularx}
\usepackage{todonotes}
\setlength{\marginparwidth}{2cm}
\usepackage[pdfpagelabels]{hyperref}
\usepackage[capitalise,noabbrev]{cleveref}
\usepackage{nameref}

% Horizontal rule
\newcommand\hr{
    \begin{center}
        \noindent\rule[0.5ex]{0.975\linewidth}{0.5pt}
    \end{center}
}


% ENVIRONMENTS
\usepackage[most]{tcolorbox}
\tcbuselibrary{theorems, breakable}
\theoremstyle{definition}

\tcbset{
    thmbox/.style={
        enhanced,
        breakable,
        sharp corners,
        borderline west={2pt}{0pt}{#1},
        boxrule=0pt,
        fonttitle=\sffamily\bfseries,
        coltitle=black,
        colframe=#1,
        colback=white!95!#1,
        colbacktitle=white!75!#1,
        bottomtitle=-0.5pt,
        parbox=false,
        before skip=1.5em,
        after skip=1.5em
    }
}

\newtcbtheorem[number within=chapter]{definition}{Definition}{
    label type=definition,
    thmbox=blue!75!white
}{def}
\newtcbtheorem[number within=chapter]{theorem}{Theorem}{
    label type=theorem,
    thmbox=red!75!white
}{thm}
\newtcbtheorem[number within=chapter]{lemma}{Lemma}{
    label type=lemma,
    thmbox=red!75!white
}{lem}
\newtcbtheorem[number within=chapter]{proposition}{Proposition}{
    label type=proposition,
    thmbox=red!75!white
}{prop}
\newtcbtheorem[number within=chapter]{example}{Example}{
    label type=ex,
    enhanced jigsaw,
    breakable,
    sharp corners,
    borderline west={2pt}{0pt}{black!75!white},
    boxrule=0pt,
    fonttitle=\sffamily\bfseries,
    coltitle=black,
    attach title to upper={\quad},
    parbox=false,
    before skip=1.5em,
    after skip=1.5em
}{ex}
\newtcolorbox{remark}{
    enhanced jigsaw,
    breakable,
    sharp corners,
    borderline west={2pt}{0pt}{black!75!white},
    boxrule=0pt,
    title=Remark,
    fonttitle=\sffamily\bfseries,
    coltitle=black,
    attach title to upper={\quad},
    parbox=false,
    before skip=1.5em,
    after skip=1.5em
}

\crefname{def}{definition}{definitions}
\crefname{thm}{theorem}{theorems}
\crefname{lem}{lemma}{lemmas}
\crefname{prop}{proposition}{propositions}
\crefname{ex}{example}{examples}

% Fix theorem spacing
\makeatletter
\def\thm@space@setup{%
  \thm@preskip=\parskip \thm@postskip=0pt
}

% Exercises
\newenvironment{exercises}
    {\begin{enumerate}[font=\sffamily\bfseries]}
    {\end{enumerate}}


% TITLE AND SECTION STYLING PACKAGES
\usepackage{titling}
\usepackage{titleps}
\usepackage{sectsty}

% Headers and fonts
\newpagestyle{main}[\sffamily]{
    \setheadrule{.7pt}%
    \sethead{\chaptername\ \thechapter}{}{\chaptertitle}
    \setfoot{}{\thepage}{}
}

% Set title page and section headings to sans serif
\pretitle{\begin{center}\LARGE\sffamily}
\posttitle{\par\end{center}\vskip 0.5em}
\preauthor{\begin{center}
    \large\sffamily \lineskip 0.5em%
    \begin{tabular}[t]{c}}
\postauthor{\end{tabular}\par\end{center}}

\allsectionsfont{\sffamily}
\renewcommand{\abstractname}{\textsf{Introduction}}


% C'est moi!
\author{Eric M. Ordo\~nez}


\title{Real Analysis}
\date{}

\begin{document}
    \maketitle
    \thispagestyle{empty}

    \begin{abstract}
        These are my course notes for Real Variables (MATH 447), which I took at the University of Illinois Urbana--Champaign as instructed by Marius Junge.
        The course is a careful development of elementary real analysis for those who intend to take graduate courses in mathematics.
        It covers, with a fair amount of abstraction and numerous proofs:
        \begin{itemize}[noitemsep]
            \item Real numbers
            \item Sequences
            \item Metric spaces
            \item Spaces of continuous functions
            \item Differentiation
            \item Integration
        \end{itemize}

        I adapted the 447 lectures and course material for personal reference.
        All credit for the exercises goes to Jason Elliot; the selected solutions are my own.
        I take all responsibility for any mistakes in the text.
        Please contact me for corrections or questions through my \href{https://ericmordonez.com}{website}.

        Last updated: \today.
    \end{abstract}

    {\sffamily\tableofcontents}
    \thispagestyle{empty}

    % Begin lectures
    \pagestyle{main}

    \part{Real Numbers}
        \chapter{Construction of the real numbers}
        This course is an introduction to analysis of the real numbers, $\R$.
        But what is $\R$?
        There are two broad answers we may be familiar with from calculus but have never had to formally define:
        \begin{enumerate}[noitemsep]
            \item $\R$ is a unique complete and ordered field;
            \item $\R$ is a particular subset of $2^{\Q}$.
        \end{enumerate}
        This chapter will connect both answers en route to a proper, mathematical definition of the real numbers.
        The plan is:
        \begin{enumerate}[noitemsep]
            \item An axiomatic definition of $\N$ followed by the Grothendieck constructions of $\Z$ and $\Q$;
            \item The Dedekind construction of $\R$ from $\Q$. 
        \end{enumerate}
    
        A purely set theoretic approach is beyond our scope, but we will utilize basic set theory and topology.
        We also assume basic knowledge of algebraic structures like groups, rings, and fields.
        The goal is a rigorous foundation of calculus emphasizing self-contained proofs that convince us that the tools and techniques we are familiar with from calculus are indeed valid.    
        
        \subfile{lecture_01.tex}
        \subfile{lecture_02.tex}
        \subfile{lecture_03.tex}
        \subfile{lecture_04.tex}
        \subfile{lecture_05.tex}
    % End lectures
\end{document}
