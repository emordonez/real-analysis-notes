% !TeX program = lualatex
% !TeX encoding = utf-8

\documentclass[master.tex]{subfiles}
% \documentclass[11pt]{report}
% % MATH PACKAGES
\usepackage{amsmath}
\usepackage{amssymb}
\usepackage{amsthm}
\usepackage{mathtools}
\usepackage{mathrsfs}
\usepackage{bm}
\usepackage{faktor}
\usepackage{nicefrac}

% Blackboard bold shortcuts
\newcommand{\N}{\mathbb{N}}
\newcommand{\Z}{\mathbb{Z}}
\newcommand{\Q}{\mathbb{Q}}
\newcommand{\R}{\mathbb{R}}
\newcommand{\C}{\mathbb{C}}

% Changes to default symbols and letters
\let\implies\Rightarrow
\let\impliedby\Leftarrow
\let\iff\Leftrightarrow
\let\epsilon\varepsilon
\let\phi\varphi
\renewcommand{\vec}[1]{\bm{#1}}
\renewcommand\qedsymbol{$\blacksquare$}

% Math operator declarations and commands
\DeclareMathOperator{\Ima}{Im}
\DeclareMathOperator{\dis}{d}
\DeclareMathOperator{\Lim}{Lim}
\DeclareMathOperator{\interior}{Int}

\DeclarePairedDelimiter\abs{\lvert}{\rvert}
\makeatletter
\let\oldabs\abs
\def\abs{\@ifstar{\oldabs}{\oldabs*}}

\DeclarePairedDelimiter\set{\{}{\}}
\makeatletter
\let\oldset\set
\def\set{\@ifstar{\oldset}{\oldset*}}

\newcommand{\eqclass}[1]{\left[ #1 \right]}
\newcommand{\mult}{{}\cdot{}}
\newcommand{\Mod}[1]{\ (\mathrm{mod}\ #1)}


% FORMATTING PACKAGES
\usepackage[left=1.5in,right=1.5in,top=1in,bottom=1in]{geometry}
% \usepackage{parskip}
\usepackage{textcomp}
\usepackage{eqparbox}
\usepackage{float}
\usepackage{graphicx}
\usepackage{xcolor}
\usepackage{enumitem}
\usepackage{emptypage}
\usepackage{microtype}
\usepackage{tabularx}
\usepackage{todonotes}
\setlength{\marginparwidth}{2cm}
\usepackage[pdfpagelabels]{hyperref}
\usepackage[capitalise,noabbrev]{cleveref}
\usepackage{nameref}

% Horizontal rule
\newcommand\hr{
    \begin{center}
        \noindent\rule[0.5ex]{0.975\linewidth}{0.5pt}
    \end{center}
}


% ENVIRONMENTS
\usepackage[most]{tcolorbox}
\tcbuselibrary{theorems, breakable}
\theoremstyle{definition}

\tcbset{
    thmbox/.style={
        enhanced,
        breakable,
        sharp corners,
        borderline west={2pt}{0pt}{#1},
        boxrule=0pt,
        fonttitle=\sffamily\bfseries,
        coltitle=black,
        colframe=#1,
        colback=white!95!#1,
        colbacktitle=white!75!#1,
        bottomtitle=-0.5pt,
        parbox=false,
        before skip=1.5em,
        after skip=1.5em
    }
}

\newtcbtheorem[number within=chapter]{definition}{Definition}{
    label type=definition,
    thmbox=blue!75!white
}{def}
\newtcbtheorem[number within=chapter]{theorem}{Theorem}{
    label type=theorem,
    thmbox=red!75!white
}{thm}
\newtcbtheorem[number within=chapter]{lemma}{Lemma}{
    label type=lemma,
    thmbox=red!75!white
}{lem}
\newtcbtheorem[number within=chapter]{proposition}{Proposition}{
    label type=proposition,
    thmbox=red!75!white
}{prop}
\newtcbtheorem[number within=chapter]{example}{Example}{
    label type=ex,
    enhanced jigsaw,
    breakable,
    sharp corners,
    borderline west={2pt}{0pt}{black!75!white},
    boxrule=0pt,
    fonttitle=\sffamily\bfseries,
    coltitle=black,
    attach title to upper={\quad},
    parbox=false,
    before skip=1.5em,
    after skip=1.5em
}{ex}
\newtcolorbox{remark}{
    enhanced jigsaw,
    breakable,
    sharp corners,
    borderline west={2pt}{0pt}{black!75!white},
    boxrule=0pt,
    title=Remark,
    fonttitle=\sffamily\bfseries,
    coltitle=black,
    attach title to upper={\quad},
    parbox=false,
    before skip=1.5em,
    after skip=1.5em
}

\crefname{def}{definition}{definitions}
\crefname{thm}{theorem}{theorems}
\crefname{lem}{lemma}{lemmas}
\crefname{prop}{proposition}{propositions}
\crefname{ex}{example}{examples}

% Fix theorem spacing
\makeatletter
\def\thm@space@setup{%
  \thm@preskip=\parskip \thm@postskip=0pt
}

% Exercises
\newenvironment{exercises}
    {\begin{enumerate}[font=\sffamily\bfseries]}
    {\end{enumerate}}


% TITLE AND SECTION STYLING PACKAGES
\usepackage{titling}
\usepackage{titleps}
\usepackage{sectsty}

% Headers and fonts
\newpagestyle{main}[\sffamily]{
    \setheadrule{.7pt}%
    \sethead{\chaptername\ \thechapter}{}{\chaptertitle}
    \setfoot{}{\thepage}{}
}

% Set title page and section headings to sans serif
\pretitle{\begin{center}\LARGE\sffamily}
\posttitle{\par\end{center}\vskip 0.5em}
\preauthor{\begin{center}
    \large\sffamily \lineskip 0.5em%
    \begin{tabular}[t]{c}}
\postauthor{\end{tabular}\par\end{center}}

\allsectionsfont{\sffamily}
\renewcommand{\abstractname}{\textsf{Introduction}}


% C'est moi!
\author{Eric M. Ordo\~nez}


\begin{document}
\section{An axiomatic definition of the natural numbers}
We begin with a nonempty set called $\N$.
We know what it should look like, but how can we precisely define it?
Assuming we know nothing about it except that it exists, we will state three axiomatic properties that it should have:
\begin{enumerate}[label=\textbf{a\arabic*)}]
    \item $1 \in \N$.\footnote{
        The standard axiomatization of $\N$ starts with $0$, but we start with $1$ to simplify our approach in constructing $\R$.
        Starting this way ultimately does not affect any of our subsequent results.
    }
    \item $n \in \N \implies (n + 1) \in \N$.
    \item If $A \subseteq \N$ is a set such that $1 \in A$ and $n \in A \implies (n + 1) \in A$, then $A = \N$.
\end{enumerate}

But what is $+$?
Remember that we know nothing about $\N$, so we cannot say right away, for example, that $1 + 1 = 2$.\footnote{
    Addition, multiplication, and other properties will be defined and proven in the exercises.
}
Instead, we generalize these axioms with an (almost) equivalent set of three other axioms.
Let $\N$ be a set containing $1$ and equipped with a function $S: \N \rightarrow \N$ such that:
\begin{enumerate}[label=\textbf{b\arabic*)}]
    \item $S(\N) = \N \setminus \set{1}$.
    \item $S$ is injective.
    \item If $A \subseteq \N$ is a set such that $1 \in A$ and $S(A) \subseteq A$, then $A = \N$.
\end{enumerate}

The formal link between these sets of axioms is the definition $S(n) := n + 1$, which we call a successor function.
The latter set gives the standard axiomatization of the natural numbers known as the Peano axioms.\footnote{
    Named in honor of Giuseppe Peano, 1858--1932.
}

\begin{definition}[title=Natural numbers, label=naturals]
    The \emph{natural numbers} $\N$ are a set equipped with a successor function $S: \N \rightarrow \N$ and that satisfies the \emph{Peano axioms:}
    \begin{enumerate}
        \item $1 \in \N$.
        \item $S(\N) = \N \setminus \set{1}$.
        \item $S$ is injective.
        \item If $A \subseteq \N$ is a set such that $1 \in A$ and $S(A) \subseteq A$, then $A = \N$.
    \end{enumerate}
\end{definition}

But do axioms \textbf{a1--3} and the Peano axioms produce the same version of $\N$?
Consider what would happen if $\N$ were a finite set.
A finite $\N$ can satisfy axioms \textbf{a1--3}, but our intuition tells us that $\N$ should be infinite.
This is further explored in the exercises.
(Note: Adding the assumption that $\N$ is infinite to axioms \textbf{a1--3} makes them equivalent to the Peano axioms).

    \subsection*{Exercises}
    \begin{exercises}
        \item Let $\N$ be the natural numbers given by the Peano axioms in \Cref{def:naturals}.
        Prove that $S$ is injective if and only if $\N$ is not a finite set.

        \item Prove that the fourth Peano axiom implies the principle of mathematical induction:\footnote{
            We will henceforth freely use induction as a proof technique.
        }
        \begin{quote}
            Let $P(n)$ be a proposition whose truth value depends on the value $n \in \N$.
            Suppose $P(1)$ is true and that $P(n) \implies P(n + 1)$ for all $n \geq 1$.
            Then $P(n)$ is true for all $n \in \N$.
        \end{quote}

        \item Define \emph{addition} $(+)$ in $\N$ by
        \begin{itemize}
            \item $\forall n \in \N \quad n + 1 = S(n)$, and
            \item $\forall m, n \in \N \quad n + S(m) = S(n + m)$;
        \end{itemize}
        and \emph{multiplication} $(\cdot)$ in $\N$ by
        \begin{itemize}
            \item $\forall n \in \N \quad n \cdot 1 = n$, and
            \item $\forall m, n \in \N \quad n \cdot S(m) = n \cdot m + n$.
        \end{itemize}
        Prove the following properties:
        \begin{exercises}
            \item $n \in \N \setminus \set{1} \implies \exists! \, n' \in \N$ such that $n = S(n')$.
            \item $\forall a \in \N \quad a + 1 = 1 + a$.
            \item $+$ is associative: $\forall a, b, c \in \N \quad a + (b + c) = (a + b) + c$.
            \item $+$ is commutative: $\forall a, b \in \N \quad a + b = b + a$.
            \item $\cdot$ distributes over $+$: $\forall a, b, c \in \N \quad a \cdot (b + c) = (a \cdot b) + (a \cdot c)$.
        \end{exercises}
    \end{exercises}
\end{document}
