% !TeX program = lualatex
% !TeX encoding = utf-8

\documentclass[master.tex]{subfiles}
% \documentclass[11pt]{report}
% % MATH PACKAGES
\usepackage{amsmath}
\usepackage{amssymb}
\usepackage{amsthm}
\usepackage{mathtools}
\usepackage{mathrsfs}
\usepackage{bm}
\usepackage{faktor}
\usepackage{nicefrac}

% Blackboard bold shortcuts
\newcommand{\N}{\mathbb{N}}
\newcommand{\Z}{\mathbb{Z}}
\newcommand{\Q}{\mathbb{Q}}
\newcommand{\R}{\mathbb{R}}
\newcommand{\C}{\mathbb{C}}

% Changes to default symbols and letters
\let\implies\Rightarrow
\let\impliedby\Leftarrow
\let\iff\Leftrightarrow
\let\epsilon\varepsilon
\let\phi\varphi
\renewcommand{\vec}[1]{\bm{#1}}
\renewcommand\qedsymbol{$\blacksquare$}

% Math operator declarations and commands
\DeclareMathOperator{\Ima}{Im}
\DeclareMathOperator{\dis}{d}
\DeclareMathOperator{\Lim}{Lim}
\DeclareMathOperator{\interior}{Int}

\DeclarePairedDelimiter\abs{\lvert}{\rvert}
\makeatletter
\let\oldabs\abs
\def\abs{\@ifstar{\oldabs}{\oldabs*}}

\DeclarePairedDelimiter\set{\{}{\}}
\makeatletter
\let\oldset\set
\def\set{\@ifstar{\oldset}{\oldset*}}

\newcommand{\eqclass}[1]{\left[ #1 \right]}
\newcommand{\mult}{{}\cdot{}}
\newcommand{\Mod}[1]{\ (\mathrm{mod}\ #1)}


% FORMATTING PACKAGES
\usepackage[left=1.5in,right=1.5in,top=1in,bottom=1in]{geometry}
% \usepackage{parskip}
\usepackage{textcomp}
\usepackage{eqparbox}
\usepackage{float}
\usepackage{graphicx}
\usepackage{xcolor}
\usepackage{enumitem}
\usepackage{emptypage}
\usepackage{microtype}
\usepackage{tabularx}
\usepackage{todonotes}
\setlength{\marginparwidth}{2cm}
\usepackage[pdfpagelabels]{hyperref}
\usepackage[capitalise,noabbrev]{cleveref}
\usepackage{nameref}

% Horizontal rule
\newcommand\hr{
    \begin{center}
        \noindent\rule[0.5ex]{0.975\linewidth}{0.5pt}
    \end{center}
}


% ENVIRONMENTS
\usepackage[most]{tcolorbox}
\tcbuselibrary{theorems, breakable}
\theoremstyle{definition}

\tcbset{
    thmbox/.style={
        enhanced,
        breakable,
        sharp corners,
        borderline west={2pt}{0pt}{#1},
        boxrule=0pt,
        fonttitle=\sffamily\bfseries,
        coltitle=black,
        colframe=#1,
        colback=white!95!#1,
        colbacktitle=white!75!#1,
        bottomtitle=-0.5pt,
        parbox=false,
        before skip=1.5em,
        after skip=1.5em
    }
}

\newtcbtheorem[number within=chapter]{definition}{Definition}{
    label type=definition,
    thmbox=blue!75!white
}{def}
\newtcbtheorem[number within=chapter]{theorem}{Theorem}{
    label type=theorem,
    thmbox=red!75!white
}{thm}
\newtcbtheorem[number within=chapter]{lemma}{Lemma}{
    label type=lemma,
    thmbox=red!75!white
}{lem}
\newtcbtheorem[number within=chapter]{proposition}{Proposition}{
    label type=proposition,
    thmbox=red!75!white
}{prop}
\newtcbtheorem[number within=chapter]{example}{Example}{
    label type=ex,
    enhanced jigsaw,
    breakable,
    sharp corners,
    borderline west={2pt}{0pt}{black!75!white},
    boxrule=0pt,
    fonttitle=\sffamily\bfseries,
    coltitle=black,
    attach title to upper={\quad},
    parbox=false,
    before skip=1.5em,
    after skip=1.5em
}{ex}
\newtcolorbox{remark}{
    enhanced jigsaw,
    breakable,
    sharp corners,
    borderline west={2pt}{0pt}{black!75!white},
    boxrule=0pt,
    title=Remark,
    fonttitle=\sffamily\bfseries,
    coltitle=black,
    attach title to upper={\quad},
    parbox=false,
    before skip=1.5em,
    after skip=1.5em
}

\crefname{def}{definition}{definitions}
\crefname{thm}{theorem}{theorems}
\crefname{lem}{lemma}{lemmas}
\crefname{prop}{proposition}{propositions}
\crefname{ex}{example}{examples}

% Fix theorem spacing
\makeatletter
\def\thm@space@setup{%
  \thm@preskip=\parskip \thm@postskip=0pt
}

% Exercises
\newenvironment{exercises}
    {\begin{enumerate}[font=\sffamily\bfseries]}
    {\end{enumerate}}


% TITLE AND SECTION STYLING PACKAGES
\usepackage{titling}
\usepackage{titleps}
\usepackage{sectsty}

% Headers and fonts
\newpagestyle{main}[\sffamily]{
    \setheadrule{.7pt}%
    \sethead{\chaptername\ \thechapter}{}{\chaptertitle}
    \setfoot{}{\thepage}{}
}

% Set title page and section headings to sans serif
\pretitle{\begin{center}\LARGE\sffamily}
\posttitle{\par\end{center}\vskip 0.5em}
\preauthor{\begin{center}
    \large\sffamily \lineskip 0.5em%
    \begin{tabular}[t]{c}}
\postauthor{\end{tabular}\par\end{center}}

\allsectionsfont{\sffamily}
\renewcommand{\abstractname}{\textsf{Introduction}}


% C'est moi!
\author{Eric M. Ordo\~nez}


\begin{document}
\section{Three key properties of the real numbers}
    We now state and prove three key properties of $\R$ that will be essential for future sections.
    \begin{enumerate}
        \item $\Q$ is dense in $\R$.\footnote{
            We will give and utilize a topological definition of density in later sections.
        }
        \item $\R$ is not countable.\footnote{
            Recall that a set is \emph{countable} if there exists a bijection between it and $\N$.
        }
        \item For all $a \in \R$, there exists $n \in \N$ such that $n > a$ (the Archimedean property).
    \end{enumerate}    

    \subsection{Density of the rationals in the reals}
    Let $D_1 < D_2 \in \R$ be Dedekind cuts.
    To prove density of $\Q$ in $\R$, it suffices to show that there exists a rational number in between $D_1$ and $D_2$.

    By definition $D_1 \subsetneq D_2$, so there exists $d \in D_2$ that is not in $D_1$.
    Then the Dedekind cut given by $\phi(d) = (-\infty, d)$ satisfies $D_1 \subsetneq \phi(d) \subsetneq D_2
    $, which implies $D_1 < \phi(d) < D_2$. \qed

    \subsection{The reals are uncountable}
    To prove this property, we first prove the uncountability of the power set of $\N$.
    \begin{lemma}{}{power-n-uncountable}
        $2^{\N}$ is uncountable.
        \hr{}
        \begin{proof}
            Assume toward contradiction that $2^{\N}$ is countable.
            Then there exists a bijection $\phi : \N \rightarrow 2^{\N}$.
            Define the set
            \[
                B := \set{n \in \N : n \notin \phi(n)} \subset \N  
            .\]
            Since $B \in 2^{\N}$ and $\phi$ is bijective, there exists $m \in \N$ such that $\phi(m) = B$.
            Then there are two possible cases:
            \begin{enumerate}
                \item $m \in B$.
                Then $m \notin \phi(m) = B$, a contradiction.

                \item $m \notin B = \phi(m)$.
                Then $m \in B$, another contradiction.
            \end{enumerate}
            Thus, $2^{\N}$ must be uncountable.
        \end{proof}
    \end{lemma}
    
    Note that if a set is countable, then every subset of that set is either countable or finite, since every subset of $\N$ is either countable or finite.
    This fact motivates the following.

    Let $\phi : 2^{\N} \rightarrow [0, 1]$ be defined by
    \[
        \phi(A) = \sum_{k \in A} 3^{-k} = \sup \bigg\{ \sum_{\substack{k \in A \\ k \leq n}} 3^{-k}: n \in \N \bigg\}
    .\]
    We will show that $\phi$ is both well-defined and injective.\footnote{
        The usual proof that $[0, 1]$ is uncountable uses binary rather than ternary expansions.
        However, such a mapping is \emph{not} injective.
        For example, take a number in the unit interval with a binary expansion whose last bit is $1$.
        Changing that bit to $0$ followed by an infinite sequence of $1$'s gives another representation of the same number.
    }
    
    For the former, we need to show that the supremum on the right-hand side exists.
    Fix $A \subset \N$.
    Then for all $n \in \N$,
    \begin{align*}
        \sum_{\substack{k \in A \\ k \leq n}} 3^{-k} &\leq \sum_{1 \leq k \leq n} 3^{-k} \\
        &= \frac{1}{3} \left(\sum_{k \leq n - 1} 3^{-k}\right) \\
        &= \frac{1}{3} \left(\frac{1 - 3^{-n}}{1 - 1/3}\right) \\
        &\leq \frac{1}{3} \cdot \frac{3}{2} \\
        &= \frac{1}{2},
    \end{align*}
    so $\phi(A)$ is bounded above and therefore has a least upper bound by the completeness of $\R$.

    For injectivity, let $A \neq B \subset \N$, and we must show that $\phi(A) \neq \phi(B)$.
    For all $n \in \N$, indicate with $1$ whether $n$ is in $A$ or $B$, and $0$ otherwise:
    \[
        \begin{array}{c|cccccc}
            n & 1 & 2 & 3 & 4 & 5 & \cdots \\
            \hline
            A & 0 & 1 & 1 & 0 & 1 & \cdots \\
            B & 0 & 1 & 0 & 0 & 1 & \cdots
        \end{array}
    \]
    Without loss of generality, let $m_0 = \min{\set{m \in A : m \notin B}}$, i.e.\ $m_0$ is the first number where the two sets disagree.
    So for every $j < m_0$, both $j \in A$ and $j \in B$.
    Then
    \[
        \phi(A) \geq 3^{-m_0} + \sum_{\substack{j < m_0 \\ j \in A}} 3^{-j}, \quad \phi(B) = \sup_k \sum_{\substack{j \leq k \\ j \in B}} 3^{-j}
    ,\]
    and for simplicity denote
    \[
        \alpha_{m_0} = \sum_{\substack{j < m_0 \\ j \in A}} 3^{-j}, \quad \phi_k(B) = \sum_{\substack{j \leq k \\ j \in B}} 3^{-j}
    .\]
    Fix $k$.
    Since $m_0 \notin B$ we have
    \[
        \phi_k(B) = \alpha_{m_0} + \sum_{\substack{m_0 < j \in B \\ j \leq k}} 3^{-j} \leq \alpha_{m_0} + \sum_{m_0 < j \leq k} 3^{-j}
    .\]
    In particular, the summation on the right-hand side yields
    \begin{align*}
        \sum_{m_0 < j \leq k} 3^{-j} &\leq \sum_{j > m_0} 3^{-j} \\
        &= 3^{-(m_0 + 1)} \sum_{j \geq 0} 3^{-j} \\
        &= 3^{-(m_0 + 1)} \cdot \frac{1}{1 - 1/3} \\
        &= 3^{-(m_0 + 1)} \cdot \frac{3}{2} \\
        &= \frac{3^{-m_0}}{2}.
    \end{align*}
    This implies
    \[
        \phi_k(B) \leq \alpha_{m_0} + \frac{3^{-m_0}}{2} < \phi(A)
    ,\]
    and therefore
    \[
        \phi(B) = \sup_k \phi_k(B) < \phi(A)
    .\]\qed
    
    This result now shows us that the real numbers are indeed uncountable.
    \begin{theorem}{}{r-uncountable}
        $\R$ is uncountable.
        \hr{}
        \begin{proof}
            Let $\phi : 2^{\N} \rightarrow [0, 1]$ be defined as before.
            Assume toward contradiction that $\R$ is countable.
            Then $[0, 1]$ must also be countable, which would imply $\phi(2^\N) \subseteq [0, 1]$ is countable.
            But $\phi$ is injective and there cannot exist an injection from an uncountable set into a countable set, so $2^{\N}$ would have to be countable in contradiction to \Cref{lem:power-n-uncountable}.
        \end{proof}
    \end{theorem}
    This theorem immediately tells us that $[0, 1]$ is uncountable, another useful result.

    \subsection{The Archimedean property}
    We wish to show that for all $a \in \R$, there exists $n \in \N$ such that $a < n$.
    Suppose otherwise by way of contradiction.
    Then $\N$ is bounded above, so $\sup{\N} = S$ exists by the completeness of $\R$.
    Then there exists natural $n$ such that $S - \frac{1}{2} < n < S$.
    Indeed, $n + 1 \leq S$ is also a natural number, but
    \[
        S + \frac{1}{2} = S - \frac{1}{2} + 1 < n + 1 \leq S
    ,\]
    a clear contradiction. \qed


    \subsection*{Exercises}
    \begin{exercises}
        \item Using the previous section's definition of the positive reals $\R^{+}$, prove the Archimedean property of $\R^{+}$:
        \[
            \forall a \in \R^{+} \: \exists n \in \N \text{ such that } n > a
        .\]
    \end{exercises}
\end{document}
