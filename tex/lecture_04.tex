% !TeX program = lualatex
% !TeX encoding = utf-8

\documentclass[master.tex]{subfiles}
% \documentclass[11pt]{report}
% % MATH PACKAGES ===============================================================
\usepackage{amsmath}
\usepackage{amssymb}
\usepackage{amsthm}
\usepackage{mathtools}
\usepackage{mathrsfs}
\usepackage{bm}
\usepackage{faktor}
\usepackage{nicefrac}

% Blackboard bold shortcuts
\newcommand{\N}{\mathbb{N}}
\newcommand{\Z}{\mathbb{Z}}
\newcommand{\Q}{\mathbb{Q}}
\newcommand{\R}{\mathbb{R}}
\newcommand{\C}{\mathbb{C}}

% Changes to default symbols and letters
\let\implies\Rightarrow
\let\impliedby\Leftarrow
\let\iff\Leftrightarrow
\let\epsilon\varepsilon
\let\phi\varphi
\renewcommand{\vec}[1]{\bm{#1}}
\renewcommand{\qedsymbol}{$\blacksquare$}

% Math operator declarations and commands
\DeclareMathOperator{\dis}{d}
\DeclareMathOperator{\Lim}{Lim}
\DeclareMathOperator{\interior}{Int}

\makeatletter
\DeclarePairedDelimiter\abs{\lvert}{\rvert}
\let\oldabs\abs
\def\abs{\@ifstar{\oldabs}{\oldabs*}}

\DeclarePairedDelimiter\setc{\{}{\}}
\let\oldset\setc
\def\setc{\@ifstar{\oldset}{\oldset*}}

\newcommand{\set}[2][\@empty]{%
    \ifx\@empty#1
        \setc{\,#2\,}
    \else
        \setc{\,#1 \;\middle|\; #2\,}
    \fi
}
\makeatother

\newcommand{\eqclass}[1]{\left[ #1 \right]}
\newcommand{\mult}{{}\cdot{}}
\newcommand{\Mod}[1]{\ (\mathrm{mod}\ #1)}


% FORMATTING PACKAGES =========================================================
\usepackage[top=1in,right=1.5in,bottom=1in,left=1.5in,asymmetric]{geometry}
\setlength{\marginparwidth}{2cm}
% \usepackage{parskip}
\usepackage{textcomp}
\usepackage{eqparbox}
\usepackage{float}
\usepackage{graphicx}
\usepackage{xcolor-material}
\usepackage{enumitem}
\usepackage{emptypage}
\usepackage{microtype}
\usepackage{tabularx}
\usepackage[pdfpagelabels]{hyperref}
\usepackage[capitalise,noabbrev]{cleveref}
\usepackage{nameref}

% Horizontal rule
\newcommand\hr{
    \begin{center}
        \noindent\rule[0.5ex]{0.975\linewidth}{0.5pt}
    \end{center}
}


% ENVIRONMENTS ================================================================
\usepackage[most]{tcolorbox}
\tcbuselibrary{breakable, theorems}

\ExplSyntaxOn
\NewDocumentCommand{\betternewtcbtheorem}{O{}mmmm}{%
    \newtcbtheorem[#1]{#2inner}{#3}{#4}{#5}
    \NewDocumentEnvironment{#2}{O{}}{%
        \keys_set:nn { hushus/tcb } { ##1 }
        \hushus_tcb_begin:nVV {#2inner} \l__hushus_tcb_title_tl \l__hushus_tcb_label_tl
    }{%
        \end{#2inner}
    }
    \cs_if_exist:cF { c@#5 } { \newcounter{#5} }
}
\cs_new_protected:Nn \hushus_tcb_begin:nnn {%
    \begin{#1}{#2}{#3}}
\cs_generate_variant:Nn \hushus_tcb_begin:nnn { nVV }
\keys_define:nn { hushus/tcb }{%
    title .tl_set:N = \l__hushus_tcb_title_tl,
    label .tl_set:N = \l__hushus_tcb_label_tl,
}
\ExplSyntaxOff

\tcbset{
    thmtitle/.style={
        attach title to upper=\\[0.5em]
    },
    remtitle/.style={
        attach title to upper=\quad
    },
    thmbox/.style 2 args={
        enhanced jigsaw,
        breakable,
        sharp corners,
        skin=bicolor,
        borderline west={3pt}{0pt}{#1},
        boxrule=0pt,
        fonttitle=\sffamily\bfseries,
        description font=\sffamily\mdseries,
        description delimiters={(}{)},
        description color=#1,
        separator sign={\ },
        coltitle=#1,
        colframe=#1,
        colback=#2,
        colbacklower=white,
        parbox=false,
        before skip=1.5em,
        after skip=1.5em
    }
}

\betternewtcbtheorem[number within=chapter]{definition}{Definition}{
    thmtitle,
    label type=definition,
    thmbox={MaterialBlue700}{MaterialBlue50},
}{def}
\betternewtcbtheorem[number within=chapter]{theorem}{Theorem}{
    thmtitle,
    label type=theorem,
    thmbox={MaterialPurple700}{MaterialPurple50}
}{thm}
\betternewtcbtheorem[number within=chapter]{lemma}{Lemma}{
    thmtitle,
    label type=lemma,
    thmbox={MaterialDeepOrange700}{MaterialDeepOrange50}
}{lem}
\betternewtcbtheorem[number within=chapter]{proposition}{Proposition}{
    thmtitle,
    label type=proposition,
    thmbox={MaterialDeepOrange700}{MaterialDeepOrange50}
}{prop}
\betternewtcbtheorem[number within=chapter]{example}{Example}{
    remtitle,
    label type=example,
    thmbox={MaterialGrey700}{MaterialGrey50}
}{ex}
\newtcolorbox{remark}{
    title=Remark,
    remtitle,
    thmbox={MaterialGreen700}{MaterialGreen50}
}

\crefname{def}{definition}{definitions}
\crefname{thm}{theorem}{theorems}
\crefname{lem}{lemma}{lemmas}
\crefname{prop}{proposition}{propositions}
\crefname{ex}{example}{examples}

% Fix theorem spacing
\makeatletter
\def\thm@space@setup{%
    \thm@preskip=\parskip \thm@postskip=0pt
}
\makeatother

% Exercises
\newenvironment{exercises}
    {\begin{enumerate}[font=\sffamily\bfseries]}
    {\end{enumerate}}


% FIGURES =====================================================================
\usepackage{caption}
\usepackage{import}
\usepackage{pdfpages}
\usepackage{transparent}
    
\newcommand{\incfig}[1]{%
    % \def\svgwidth{\columnwidth}
    \import{../figures/}{#1.pdf_tex}
}
\captionsetup{
    format=plain,
    labelsep=quad,
    labelfont={sf,bf},
    font=sf
}
\numberwithin{figure}{chapter}


% TITLE AND SECTION STYLING PACKAGES ====================================================
\usepackage{etoolbox}
\usepackage{titling}
\usepackage{titleps}
\usepackage{sectsty}

% Headers
\newpagestyle{main}[\sffamily]{
    \sethead
    [\chaptername\ \thechapter][\chaptertitle][\bfseries\thepage]%
    {\S\thesection}{\sectiontitle}{\bfseries\thepage}
}
\patchcmd{\tableofcontents}{plain}{empty}{}{}
\patchcmd{\part}{plain}{empty}{}{}
\patchcmd{\chapter}{plain}{empty}{}{}

% Sans serif titles and section headings
\pretitle{\begin{center}\LARGE\sffamily}
\posttitle{\par\end{center}\vskip 0.5em}
\preauthor{\begin{center}
    \large\sffamily \lineskip 0.5em%
    \begin{tabular}[t]{c}}
\postauthor{\end{tabular}\par\end{center}}

\allsectionsfont{\sffamily}
\sectionfont{\sffamily\sectionrule{0pt}{0pt}{-1ex}{1pt}}
\renewcommand{\abstractname}{\textsf{Introduction}}


% TABLE OF CONTENTS ===========================================================
% \usepackage{tocloft}
% \renewcommand{\cfttoctitlefont}{\sffamily}
% \renewcommand{\cftpartfont}{\sffamily}
% \renewcommand{\cftpartpagefont}{\sffamily}
% \renewcommand{\cftchapfont}{\sffamily}
% \renewcommand{\cftchappagefont}{\sffamily}
% \renewcommand{\cftsecfont}{\sffamily}
% \renewcommand{\cftsecpagefont}{\sffamily}
% \renewcommand{\cftsubsecfont}{\sffamily}
% \renewcommand{\cftsubsecpagefont}{\sffamily}


% C'est moi!
\author{Eric M. Ordo\~nez}


\begin{document}
\section{Real numbers as Dedekind cuts}
Now we can construct $\R$ in a way that shows it to be both a complete, ordered field and a subset of $2^{\Q}$ by using \emph{Dedekind cuts}.

\begin{definition}{Dedekind cut}{dedekind-cut}
    A \textbf{Dedekind cut} is a set $A \subseteq \Q$ such that:
    \begin{itemize}
        \item $\forall a \in A \quad b \leq a \implies b \in A$.
        \item $\forall a \in A \quad \exists a' > a$ such that $a' \in A$.
    \end{itemize}
\end{definition}

The first condition says a Dedekind cut is open downward, i.e.\ it continues to $-\infty$.
The second condition says it is open upward, i.e.\ it has no maximal element in the cut.
(Note that this does not necessarily mean it goes to $\infty$).

\begin{example}{}{r-dedekind}
    Let $R := \set{D \subseteq \Q : D \text{ is Dedekind}}$ and define $\phi : \Q \rightarrow R$ by $q \mapsto \set{a : a < q}$.
    Then $R$ is Dedekind.
\end{example}

For a visual interpretation of Example \ref{ex:r-dedekind} (and setting aside for a moment that we do not know what ``real numbers'' are yet), imagine cutting the real line at every rational $q$.
Every rational to the left of $q$ is in the cut $D$, and no matter which $a \in D$ we pick, we can always find another rational to the right of $a$ that is arbitrarily less than $q$ and therefore in $D$.

\begin{example}{}{sqrt-2-dedekind}
    $A_{\sqrt{2}} := \set{x \in \Q : x^2 < 2 \text{ or } x \leq 0}$ is Dedekind.
\end{example}

Try showing that these example sets satisfy Definition \ref{def:dedekind-cut}.

Since $\Q$ is totally ordered, Example \ref{ex:r-dedekind} suggests that we can also order Dedekind cuts.
Indeed, this shows us that that set $R$ of Dedekind cuts is complete.

\begin{definition}{}{}
    Let $D_1, D_2 \in R$.
    If $D_1 \subseteq D_2$, then we say $D_1 \leq D_2$.
\end{definition}

\begin{lemma}{}{set-of-dedekind-complete}
    $R$ is complete.
    \hr{}
    \begin{proof}
        Let $S \subset R$ be a nonempty set of Dedekind cuts and let $D \in R$ be an upper bound for $S$.
        Then for every $s \in S$, by definition $s \leq D$ and $s \subseteq D$.

        Now define $\sup{S} := \bigcup_{s \in S} s$.
        This union is trivially contained in $D$, and we can also confirm it is in $R$ by showing it verifies the conditions of Definition \ref{def:dedekind-cut}.
        Thus, $\sup{S} \leq D$.

        To show that $\sup{S}$ is indeed a supremum, we must show that it is the least upper bound for $S$.
        Let $D' \in R$ be another upper bound.
        Then for every $s \in S$ we have $s \subseteq D'$, which implies $\sup{S} = \cup_{s \in S} s \subseteq D'$ and therefore $\sup{S} \leq D'$.
    \end{proof}
\end{lemma}

However, is $R$ the real numbers that we are familiar with?
Not quite, because this set is too big!
Note that $\Q$ itself is a Dedekind cut in $R$ that would correspond to $\infty$, which we know is not a number.
Hence, we refine our definition of $R$ to give a proper definition of the \emph{real numbers}.

\begin{definition}{Real numbers}{}
    The \textbf{real numbers} $\R$ are the set of proper Dedekind cuts
    \[
        \R := \set{D \subset \Q : D \neq \Q \text{ is Dedekind}}
    .\]
    If $D \neq \Q$, then take any $q \notin D$.
    No $q' > q$ can be in $D$, so we can equivalently define $\R$ by
    \[
       \R := \set{D \in R : \exists q \in \Q \text{ such that } D < \phi(q)}
    ,\]
    where $\phi : \Q \rightarrow \R$ is analagously defined as in Example \ref{ex:r-dedekind}.
\end{definition}

\begin{remark}
    $\R$ is complete.
    \hr{}
    \begin{proof}
        If $D \in \R$, then by definition there exists $q_0$ such that $D < \phi(q_0)$.
        This implies
        \[
            \sup{D} < \phi(q_0) \implies \sup{D} \neq \infty \implies \sup{D} \in \R
        .\]
        The rest of the proof is the same as for Lemma \ref{lem:set-of-dedekind-complete}.
    \end{proof}
\end{remark}

\hr{}

We now consider the field nature of $\R$.
Define the \textbf{positive reals} $\R^{+}$ as the set
\[
    \R^{+} := \set{D \in \R : (-\infty, 0] \subseteq D}
,\]
and define addition and multiplication in it by
\begin{align*}
    D_1 + D_2 &= \bigcup \set{d_1 + d_2 : D_1^{+}, d_2 \in D_2^{+}} \\
    D_1 D_2 &= \bigcup \set{d_1 d_2 : d_1 \in D_1^{+}, d_2 \in D_2^{+}}.
\end{align*}
Since negative numbers complicate multiplication, we can assume that $D \in \R^{+}$ is a subset of $\Q^{+}$ and ``starts'' at $0$ rather than $-\infty$.
More formally, we can view this as $\R^{+} = \set{D \subset \Q^{+} : D \cup (-\infty, 0] \text{ is Dedekind}}$.


With these operations, $\R^{+}$ satisfies the field axioms with associative, commutative, and distributive properties (verify these for practice!):
\begin{align*}
    D_1 + (D_2 + D_3) &= (D_1 + D_2) + D_3 \\
    D_1 (D_2 + D_3) &= D_1 D_2 + D_1 D_3 \\
    &\vdotswithin{=}
\end{align*}
Indeed, $\R^{+}$ is a commutative and cancellative semigroup under $+$.
Our final answer to what the real numbers are is that $\R$ is the Grothendieck group for $\R^{+}$:
\[
    \R = G(\R^{+}, +)
.\]

\begin{lemma}{}{}
    $\R$ is totally ordered.
    \hr{}
    \begin{proof}
        Let $D_1, D_2 \in \R$.
        There are two cases:
        \begin{enumerate}
            \item[] Case 1: $D_1 \subseteq D_2$.
            Then by definition $D_1 \leq D_2$, and we are done.

            \item[] Case 2: $D_1 \not\subseteq D_2$.
            Then there exists $d_1 \in D_1$ that is not in $D_2$.
            This means for every $d_2 \in D_2$ we have $d_2 < d_1$.
            By definition, this implies $D_2 \subseteq D_1$ and hence $D_2 \leq D_1$.
        \end{enumerate}
    \end{proof}
\end{lemma}

    \subsection*{Exercises}
    \begin{enumerate}[font=\bfseries]
        \item Let $\phi : \Q^{+} \rightarrow \R^{+}$ (where $\R^{+}$ is a set of Dedekind cuts) be the function defined by
        \[
            \phi(q) := \set{a \in \Q : a < q}
        .\]
        \begin{enumerate}[font=\bfseries]
            \item Prove that $\phi$ is injective and order-preserving, i.e.\ show that if $q < q'$, then $\phi(q) < \phi(q')$.
            \item Prove that $\phi$ respects both field operations:
            \[
                \phi(qr) = \phi(q) \phi(r) \text{ and } \phi(q + r) = \phi(q) + \phi(r)
            .\] 
        \end{enumerate}

        \item Let $A_{\sqrt{2}}$ be the Dedekind cut from Example \ref{ex:sqrt-2-dedekind}.
        Prove that $\sup{A_{\sqrt{2}}} = \sqrt{2}$.
    \end{enumerate}
\end{document}
