% !TeX program = lualatex
% !TeX encoding = utf-8

\documentclass[master.tex]{subfiles}
% \documentclass[11pt]{report}
% % MATH PACKAGES ===============================================================
\usepackage{amsmath}
\usepackage{amssymb}
\usepackage{amsthm}
\usepackage{mathtools}
\usepackage{mathrsfs}
\usepackage{bm}
\usepackage{faktor}
\usepackage{nicefrac}

% Blackboard bold shortcuts
\newcommand{\N}{\mathbb{N}}
\newcommand{\Z}{\mathbb{Z}}
\newcommand{\Q}{\mathbb{Q}}
\newcommand{\R}{\mathbb{R}}
\newcommand{\C}{\mathbb{C}}

% Changes to default symbols and letters
\let\implies\Rightarrow
\let\impliedby\Leftarrow
\let\iff\Leftrightarrow
\let\epsilon\varepsilon
\let\phi\varphi
\renewcommand{\vec}[1]{\bm{#1}}
\renewcommand{\qedsymbol}{$\blacksquare$}

% Math operator declarations and commands
\DeclareMathOperator{\dis}{d}
\DeclareMathOperator{\Lim}{Lim}
\DeclareMathOperator{\interior}{Int}

\makeatletter
\DeclarePairedDelimiter\abs{\lvert}{\rvert}
\let\oldabs\abs
\def\abs{\@ifstar{\oldabs}{\oldabs*}}

\DeclarePairedDelimiter\setc{\{}{\}}
\let\oldset\setc
\def\setc{\@ifstar{\oldset}{\oldset*}}

\newcommand{\set}[2][\@empty]{%
    \ifx\@empty#1
        \setc{\,#2\,}
    \else
        \setc{\,#1 \;\middle|\; #2\,}
    \fi
}
\makeatother

\newcommand{\eqclass}[1]{\left[ #1 \right]}
\newcommand{\mult}{{}\cdot{}}
\newcommand{\Mod}[1]{\ (\mathrm{mod}\ #1)}


% FORMATTING PACKAGES =========================================================
\usepackage[top=1in,right=1.5in,bottom=1in,left=1.5in,asymmetric]{geometry}
\setlength{\marginparwidth}{2cm}
% \usepackage{parskip}
\usepackage{textcomp}
\usepackage{eqparbox}
\usepackage{float}
\usepackage{graphicx}
\usepackage{xcolor-material}
\usepackage{enumitem}
\usepackage{emptypage}
\usepackage{microtype}
\usepackage{tabularx}
\usepackage[pdfpagelabels]{hyperref}
\usepackage[capitalise,noabbrev]{cleveref}
\usepackage{nameref}

% Horizontal rule
\newcommand\hr{
    \begin{center}
        \noindent\rule[0.5ex]{0.975\linewidth}{0.5pt}
    \end{center}
}


% ENVIRONMENTS ================================================================
\usepackage[most]{tcolorbox}
\tcbuselibrary{breakable, theorems}

\ExplSyntaxOn
\NewDocumentCommand{\betternewtcbtheorem}{O{}mmmm}{%
    \newtcbtheorem[#1]{#2inner}{#3}{#4}{#5}
    \NewDocumentEnvironment{#2}{O{}}{%
        \keys_set:nn { hushus/tcb } { ##1 }
        \hushus_tcb_begin:nVV {#2inner} \l__hushus_tcb_title_tl \l__hushus_tcb_label_tl
    }{%
        \end{#2inner}
    }
    \cs_if_exist:cF { c@#5 } { \newcounter{#5} }
}
\cs_new_protected:Nn \hushus_tcb_begin:nnn {%
    \begin{#1}{#2}{#3}}
\cs_generate_variant:Nn \hushus_tcb_begin:nnn { nVV }
\keys_define:nn { hushus/tcb }{%
    title .tl_set:N = \l__hushus_tcb_title_tl,
    label .tl_set:N = \l__hushus_tcb_label_tl,
}
\ExplSyntaxOff

\tcbset{
    thmtitle/.style={
        attach title to upper=\\[0.5em]
    },
    remtitle/.style={
        attach title to upper=\quad
    },
    thmbox/.style 2 args={
        enhanced jigsaw,
        breakable,
        sharp corners,
        skin=bicolor,
        borderline west={3pt}{0pt}{#1},
        boxrule=0pt,
        fonttitle=\sffamily\bfseries,
        description font=\sffamily\mdseries,
        description delimiters={(}{)},
        description color=#1,
        separator sign={\ },
        coltitle=#1,
        colframe=#1,
        colback=#2,
        colbacklower=white,
        parbox=false,
        before skip=1.5em,
        after skip=1.5em
    }
}

\betternewtcbtheorem[number within=chapter]{definition}{Definition}{
    thmtitle,
    label type=definition,
    thmbox={MaterialBlue700}{MaterialBlue50},
}{def}
\betternewtcbtheorem[number within=chapter]{theorem}{Theorem}{
    thmtitle,
    label type=theorem,
    thmbox={MaterialPurple700}{MaterialPurple50}
}{thm}
\betternewtcbtheorem[number within=chapter]{lemma}{Lemma}{
    thmtitle,
    label type=lemma,
    thmbox={MaterialDeepOrange700}{MaterialDeepOrange50}
}{lem}
\betternewtcbtheorem[number within=chapter]{proposition}{Proposition}{
    thmtitle,
    label type=proposition,
    thmbox={MaterialDeepOrange700}{MaterialDeepOrange50}
}{prop}
\betternewtcbtheorem[number within=chapter]{example}{Example}{
    remtitle,
    label type=example,
    thmbox={MaterialGrey700}{MaterialGrey50}
}{ex}
\newtcolorbox{remark}{
    title=Remark,
    remtitle,
    thmbox={MaterialGreen700}{MaterialGreen50}
}

\crefname{def}{definition}{definitions}
\crefname{thm}{theorem}{theorems}
\crefname{lem}{lemma}{lemmas}
\crefname{prop}{proposition}{propositions}
\crefname{ex}{example}{examples}

% Fix theorem spacing
\makeatletter
\def\thm@space@setup{%
    \thm@preskip=\parskip \thm@postskip=0pt
}
\makeatother

% Exercises
\newenvironment{exercises}
    {\begin{enumerate}[font=\sffamily\bfseries]}
    {\end{enumerate}}


% FIGURES =====================================================================
\usepackage{caption}
\usepackage{import}
\usepackage{pdfpages}
\usepackage{transparent}
    
\newcommand{\incfig}[1]{%
    % \def\svgwidth{\columnwidth}
    \import{../figures/}{#1.pdf_tex}
}
\captionsetup{
    format=plain,
    labelsep=quad,
    labelfont={sf,bf},
    font=sf
}
\numberwithin{figure}{chapter}


% TITLE AND SECTION STYLING PACKAGES ====================================================
\usepackage{etoolbox}
\usepackage{titling}
\usepackage{titleps}
\usepackage{sectsty}

% Headers
\newpagestyle{main}[\sffamily]{
    \sethead
    [\chaptername\ \thechapter][\chaptertitle][\bfseries\thepage]%
    {\S\thesection}{\sectiontitle}{\bfseries\thepage}
}
\patchcmd{\tableofcontents}{plain}{empty}{}{}
\patchcmd{\part}{plain}{empty}{}{}
\patchcmd{\chapter}{plain}{empty}{}{}

% Sans serif titles and section headings
\pretitle{\begin{center}\LARGE\sffamily}
\posttitle{\par\end{center}\vskip 0.5em}
\preauthor{\begin{center}
    \large\sffamily \lineskip 0.5em%
    \begin{tabular}[t]{c}}
\postauthor{\end{tabular}\par\end{center}}

\allsectionsfont{\sffamily}
\sectionfont{\sffamily\sectionrule{0pt}{0pt}{-1ex}{1pt}}
\renewcommand{\abstractname}{\textsf{Introduction}}


% TABLE OF CONTENTS ===========================================================
% \usepackage{tocloft}
% \renewcommand{\cfttoctitlefont}{\sffamily}
% \renewcommand{\cftpartfont}{\sffamily}
% \renewcommand{\cftpartpagefont}{\sffamily}
% \renewcommand{\cftchapfont}{\sffamily}
% \renewcommand{\cftchappagefont}{\sffamily}
% \renewcommand{\cftsecfont}{\sffamily}
% \renewcommand{\cftsecpagefont}{\sffamily}
% \renewcommand{\cftsubsecfont}{\sffamily}
% \renewcommand{\cftsubsecpagefont}{\sffamily}


% C'est moi!
\author{Eric M. Ordo\~nez}


\begin{document}
\subsection{Integers and rationals as Grothendieck groups}
    The Grothendieck lemma states that for any commutative semigroup with cancellation, $(S, +)$, we can construct a Groenthdieck group $G(S)$ for $S$ by defining
    \[
        G(S) := \faktor{(S \times S)}{\sim}
    \]
    with the equivalence relation
    \[
        (a, b) \sim (c, d) \iff a + d = b + c
    \]
    for all $a, b, c, d \in S$.
    
    We already showed $(\N, +)$ to be a commutative semigroup with cancellation and are naturally interested in what its Grothendieck group would be.
    However, we also showed that multiplication commutes in $\N$, and in fact, $(\N, \mult{})$ is also a commutative and cancellative semigroup.

    \begin{remark}
        Define the order relation $<$ on $\N$ by saying $n < n'$ if there exists $k \in \N$ such that $n' = n + k$.
        This relation satisfies the \emph{trichotomy law}: For any $n, n' \in \N$, exactly one of the following holds:
        \begin{enumerate}
            \item $n < n'$
            \item $n' < n$
            \item $n = n'$
        \end{enumerate}
    \end{remark}

    \begin{lemma}{}{n-multiplicative-cancellation}
        $(\N, \mult{})$ has the cancellation property.
        \hr{}
        \begin{proof}
            Let $B(m)$ be the statement
            \[
                \forall n, n' \in \N \quad nm = n'm \implies n = n'  
            \]
            for all $m \in \N$.
            We will prove this by way of another statement.

            By the trichotomy law, $n \neq n'$ if and only if $n < n'$ or $n' < n$.
            Let $B_0(m)$ be the statement
            \[
                  \forall n, n' \in \N \quad n < n' \implies nm < n'm
            ,\]
            which we prove true for all $m \in \N$.
            The base case is trivially true, so assume $B_0(m)$ is true for some $m \geq 1$.
            Then
            \begin{align*}
                n < n' &\implies nm < n'm \\
                &\implies nm + n < n'm + n' \\
                &\implies n(m + 1) < n'(m + 1).
            \end{align*}
            Returning to $B(m)$, fix $n, n' \in \N$ so that $nm = n'm$.
            Assume toward contradiction that $n \neq n'$.
            Then either
            \begin{enumerate}
                \item $n < n' \xRightarrow{B_0(m)} nm < nm'$, a contradiction; or
                \item $n' < n \xRightarrow{B_0(m)} n'm < nm$, another contradiction.
            \end{enumerate}
            This shows that $B(m)$ is true whenever $B_0(m)$ is true, which by induction is for all $m \in \N$.
        \end{proof}
    \end{lemma}

    We are now ready for a formal definition of the integers, $\Z$.
    The \textbf{integers} are the smallest abelian group containing $\N$ equipped with addition; $\Z$ is the Grothendieck group for $(\N, +)$ given by
    \begin{align*}
        \Z &:= G(\N, +) \\
        &= \faktor{(\N \times \N)}{\sim} \\
        &= \left( \N \times \set{+} \right) \cup \left( \N \times \set{-} \right) \cup \set{0},
    \end{align*}
    where each integer is an equivalence class denoted by
    \[
        \eqclass{(a, b)} \mapsto \begin{cases}
            \left( a - b, + \right) &\mbox{if $a > b$} \\
            \left( b - a, - \right) &\mbox{if $a < b$} \\
            \hfil 0 &\mbox{if $a = b$}.
        \end{cases}  
    \]
    If we set $k = a - b$, we see the first two cases correspond to the familiar notation of $k$ and $-k$, respectively.
    We will not go through this much rigor for notation in the next Grothendieck constructions; the point is to emphasize this algebraic foundation for our construction of $\R$. 
    
    So what Grothendieck groups can be constructed from $(\N, \mult{})$ and $\Z$?
    The answer, perhaps unsurprisingly, is the \emph{rational numbers}.
    \begin{definition}{Rational numbers}{rational-numbers}
        The \textbf{positive rationals} are defined by
        \[
            \Q^{+} := G(\N, \mult{}) = \set{\frac{p}{q} : p, q \in \N}
        .\]
        The \textbf{rationals} can be equivalently defined by two Grothendieck groups:
        \begin{align*}
            \Q &:= G(\Q^{+}, +) \quad \text{where} \quad \frac{p}{q} + \frac{p'}{q'} = \frac{pq' + p'q}{qq'}
            \intertext{and}
            \Q &:= G(\Z \setminus \set{0}, \mult{}) = \set{\frac{p}{q} : p \in \Z, q \in \N}.
        \end{align*}
    \end{definition}

    Like $\N$ and $\Z$, $\Q$ is a totally ordered set\footnote{
        See the exercises for further information.
        From now on, we can also relax the total order $\leq$ to the partial order $<$ defined in this subsection's remark, and $\leq$ and $<$ will behave as expected.
    } where
    \[
        \frac{p}{q} \leq \frac{p'}{q'} \iff pq' \leq p'q 
    .\]
    Moreover, $\Q$ is a canonical example of an \emph{ordered field}, an algebraic structure consisting of a set equipped with two operations, addition $+$ and multiplication $\mult{}$, and an order relation $\leq$ that satisfy the following axioms for all $a, b, c \in \Q$:\footnote{
        We should still prove that $\Q$ satisfies these axioms, but this task is left to the reader.
    }
    
    \begin{figure*}[ht]
        \begin{tabularx}{\textwidth}{p{0.45\linewidth} p{0.45\linewidth}}
            \textbf{Field axioms} & \textbf{Order axioms} \\
            \begin{itemize}[topsep=0pt]
                \item $(\Q, +, 0)$ is an abelian group.
                \item $(\Q \setminus \set{0}, \mult{}, 1)$ is an abelian group.
                \item $a \cdot (b + c) = a \cdot b + a \cdot c$.
            \end{itemize} &
            \begin{itemize}[topsep=0pt]
                \item $a \leq b \implies a + c \leq b + c$.
                \item $c > 0, \, a \leq b \implies ac \leq bc$.
                \item Either $a \leq b$ or $b \leq a$.
            \end{itemize}
        \end{tabularx}
    \end{figure*}

    More generally, ordered fields allow us to introduce the concept of \emph{upper bounds} and \emph{least upper bounds}.
    \begin{definition}{(Least) upper bound}{lub}
        Let $F$ be an ordered field.
        Given $A \subseteq F$ and $b \in F$:
        \begin{enumerate}
            \item $b$ is an \textbf{upper bound} for $A$ if $a \leq b$ for all $a \in A$.
            This is also denoted by $A \leq b$.
            \item $b$ is a \textbf{least upper bound} if $A \leq b$ and $b \leq b'$ for any upper bound $b'$.
            This is called the \textbf{supremum} of $A$, denoted by $\sup{A}$.
        \end{enumerate}
    \end{definition}
    The supremum will be key to our discussion of completeness, which is a concept we introduce here for ordered fields.
    \begin{definition}{Complete ordered field}{complete-ordered-field}
        An ordered field $F$ is called \textbf{complete} if for every subset $A \subset F$ there exists $b' \in F$ such that $A \leq b'$, then there exists $b \in F$ such that $b = \sup{A}$.
        In other words, every set $A$ bounded above has a supremum $b$.
    \end{definition}

    \begin{example}{}{}
        $\set{x \in \Q : x^2 < 2}$ has no supremum in $\Q$, so $\Q$ is not complete.
    \end{example}

    Recall from the start of the chapter our prior definitions of $\R$ as a unique complete and ordered field as well as a particular subset of $2^{\Q}$.
    Now that we have working definitions of both complete ordered fields and $\Q$, we can properly construct $\R$.


\subsection*{Exercises}
\begin{enumerate}
    \item A \emph{totally ordered semigroup} is a semigroup $(A, +)$ with an order relation $\leq$ that satisfies the following properties:
    \begin{itemize}
        \item $\forall a \in A \quad a \leq a$;
        \item $\forall a, b \in A \quad a \leq b, \ b \leq a \implies a = b$;
        \item $\forall a, b, c \in A \quad a \leq b, \ b \leq c \implies a \leq c$;
        \item Either $a \leq b$ or $b \leq a$ for all $a, b \in A$;
        \item $\forall a, b, c \in A \quad a \leq b \implies a + c \leq b + c$.
    \end{itemize}
    \begin{enumerate}
        \item Give a definition for a total order $\leq$ on $\N$ such that $n \leq S(n)$ for all $n$.
        Prove that the above properties hold.
        \item Prove that if $x \leq y$, then there exists a unique $z \in \N$ such that $x + z = y$.
    \end{enumerate}

    \item In this exercise, we finish the proof of the Grothendieck lemma.
    Let $S$ be a commutative semigroup with cancellation and $G(S)$ its Grothendieck group.
    \begin{enumerate}
        \item Prove that the function $\iota : S \rightarrow G(S)$ defined by $s \mapsto \eqclass{(s + s, s)}$ is injective and satisfies $\iota(s + t) = \iota(s) + \iota(t)$ for all $s, t \in S$.\footnote{
            This shows that $\iota$ is an injective homomorphism that embeds $S$ into $G(S)$, hence $G(S)$ ``contains'' $S$ as $\iota(S)$.
        }
        \item Let $(G, \mult{})$ be a group and $\phi : S \rightarrow G$ a function such that $\phi(x + y) = \phi(x) \cdot \phi(y)$.
        Prove that there is a function $\psi : G(S) \rightarrow G$ such that $\psi \circ \iota = \phi$ and $\psi(g + h) = \psi(g) \cdot \psi(h)$ for all $g, h \in G(S)$.\footnote{
            This shows that $G(S)$ is the ``smallest'' abelian group containing $S$.
        }
    \end{enumerate}

    \item Let $(S, +, \leq)$ be a totally ordered commutative semigroup with cancellation.
    Define an order relation $\leq$ on its Grothendieck group $G(S)$ by
    \[
        \eqclass{(a, b)} \leq \eqclass{(c, d)} \iff a + d \leq b + c
    .\]
    Prove that
    \begin{enumerate}
        \item $\forall s, t, c \in S \quad s + c \leq t + c \implies s \leq t$;
        \item $\forall s,t \in S \quad \eqclass{(s + s, s)} \leq \eqclass{(t + t, t)} \iff s \leq t$;
        \item $(G(S), \leq)$ is a totally ordered group.
    \end{enumerate}
    
    \item Given the Grothendieck group definition of $\Z$, define addition in $\Z$ as follows:
    \begin{align*}
        0 + z = z + 0 &= z \quad \forall z \in \Z \\
        (x, +) + (y, +) &= (x + y, +) \\
        (x, -) + (y, -) &= (x + y, -) \\
        (x, +) + (y, -) &=
        \begin{cases}
            (x - y, +) &\text{if } y < x \\
            (y - x, -) &\text{if } x < y \\
            \hfil 0 &\text{if } x = y.
        \end{cases}
    \end{align*}
    \begin{enumerate}
        \item Prove that the function $\phi : G(\N, +) \rightarrow \Z$ defined by
        \[
            \phi\left(\eqclass{a, b}\right) =
            \begin{cases}
                (a - b, +) &\text{if } b < a \\
                (b - a, -) &\text{if } a < b \\
                \hfil 0 &\text{if } a = b
            \end{cases}
        \]
        is bijective and satisfies $\phi\left(\eqclass{a + c, b + d}\right) = \phi\left(\eqclass{a, b}\right) + \phi\left(\eqclass{c, d}\right)$.

        \item For $x, y \in \Z$, define $x \leq y$ to mean $\phi^{-1}(x) \leq \phi^{-1}(y)$.
        Prove that $(a, +) > 0$ and $(a, -) < 0$ for all $a \in \N$.
    \end{enumerate}

    \item Define multiplication in $\Z$ as follows:
    \begin{align*}
        0 \cdot z = z \cdot 0 &= 0 \quad \forall z \in \Z \\
        (x, +) \cdot (y, +) &= (x \cdot y, +) \\
        (x, -) \cdot (y, -) &= (x \cdot y, +) \\
        (x, +) \cdot (y, -) &= (x \cdot y, -) \\
        (x, -) \cdot (y, +) &= (x \cdot y, -)
    \end{align*}
    \begin{enumerate}
        \item Prove that $(\Z \setminus \set{0}, \mult{})$ is a commutative semigroup with cancellation.
        \item $(\Z, +)$ is a totally ordered group, so verify that $a, b > 0 \implies ab > 0$.
    \end{enumerate}
\end{enumerate}
\end{document}
